\documentclass[letter, 10pt]{article}
\usepackage{fullpage}
\usepackage[margin=0.5in]{geometry}
\usepackage{graphicx}
\usepackage{caption}
\usepackage{subcaption}
\usepackage{listings}
\usepackage{amsmath}
\usepackage{float}
\usepackage{media9}
\usepackage{animate}

\pagenumbering{gobble}

\begin{document}
\noindent
\large \textbf{Rahul Ghosh} \hfill \textbf{Assignment\#2}\\
\normalsize Student ID: 5476965 \hfill CSci 5561\\

\section*{IMAGE REGISTRATION}
\subsection*{Methods}
Given an object/template, the aim of this project is to track the object/template in the given target frames.
\begin{figure}[!h]
    \centering
    \includegraphics[width=\linewidth]{HW2/RESULT/Frames.fig}
    \figure{Input Images}
    
\begin{equation*}
    filter\_x = \begin{bmatrix}
    1 & 0 & -1\\
    1 & 0 & -1\\
    1 & 0 & -1\\
    \end{bmatrix}
    and\ filter\_y = \begin{bmatrix}
    1 & 1 & 1\\
    0 & 0 & 0\\
    -1 & -1 & -1\\
    \end{bmatrix}
\end{equation*}
The gradient magnitude and direction are then calculated from the filtered images along the x and y directions as shown in Figure 1. The image is divided into blocks and the gradients in each block are divided into 6 different bins according to their gradient angle. Following are the bins used in this assignment:
\begin{equation*}
    bin_1 = [\frac{11\pi}{12}, \pi)\cup[0, \frac{\pi}{12}),
    bin_2 = [\frac{\pi}{12}, \frac{3\pi}{12}),
    bin_3 = [\frac{3\pi}{12}, \frac{5\pi}{12}),
    bin_4 = [\frac{5\pi}{12}, \frac{7\pi}{12}),
    bin_5 = [\frac{7\pi}{12}, \frac{9\pi}{12}),
    bin_6 = [\frac{9\pi}{12}, \frac{11\pi}{12})
\end{equation*}
Next, to take into account the illumination and contrast, the histograms of gradients of each block is normalized using its $2\times2$ neighbors and the corresponding histograms are concatenated resulting in a 24 ($6\times4$, one for each block) length hog descriptor for each block.

To visualize, we take the sum of the magnitude for each bin and take the mid point of the bin as the angle as shown below:
\begin{equation*}
    \theta_1 = 0, \theta_2 = \frac{2\pi}{12}, \theta_3 = \frac{4\pi}{12}, \theta_4 = \frac{6\pi}{12}, \theta_5 = \frac{8\pi}{12}, \theta_6 = \frac{10\pi}{12}
\end{equation*}

The vector is plotted perpendicular to the given block centre using the $quiver$ method in MATLAB as shown in Figure 2. 
\subsection*{RESULTS}
\includemedia[
  width=0.4\linewidth,
  height=0.3\linewidth,
  activate=pageopen,
  addresource=HW2/RESULT/Frames.MP4,
  flashvars={source=HW2/RESULT/Frames.MP4}
]{}{VPlayer.swf}

\begin{figure}[!h]
    \minipage{0.5\textwidth}
        \centering
        \includegraphics[width=\linewidth]{HW1/RESULT/GRADIENT.png}
        \subcaption{Gradients}
    \endminipage\hfill
    \minipage{0.25\textwidth}
        \centering
        \includegraphics[width=0.7\linewidth]{HW1/RESULT/GRADIENT_EYE.png}
        \subcaption{Gradients\_EYE}
    \endminipage\hfill
    \minipage{0.25\textwidth}
        \centering
        \includegraphics[width=0.7\linewidth]{HW1/RESULT/GRADIENT_TONGUE.png}
        \subcaption{Gradients\_TONGUE}
    \endminipage\hfill
    \caption{Gradients of Image}
\end{figure}

\end{document}